\documentclass[a4paper,12pt]{article}

\usepackage{amsfonts, amsmath, amssymb, authblk, scrextend, hyperref, enumerate,mathtools, tikz, csquotes, mathrsfs, lmodern,arydshln, xypic, bbold, graphicx, setspace}
\usepackage[mathcal]{eucal}
\usetikzlibrary{matrix}
\usepackage[lmargin=2.5 cm,rmargin=2.5cm,tmargin=3cm,bmargin=3cm]{geometry}

\allowdisplaybreaks
\everymath{\displaystyle}
\setlength\parindent{0pt}
\setlength\parskip{7.5pt}



\newcommand*\encircled[1]{\tikz[baseline=(char.base)]{
            \node[shape=circle,draw,inner sep=1pt] (char) {#1};}}
\newcommand*\rected[1]{\tikz[baseline=(char.base)]{
            \node[shape=rectangle,draw,inner sep=2pt] (char) {#1};}}
\newcommand{\dlim}{\underset{\longrightarrow}{\lim} \ }
\newcommand{\field}[1]{\mathbb{#1}}
\newcommand{\N}{\field{N}}
\newcommand{\one}{\field{1}}
\newcommand{\disp}{\displaystyle}
\newcommand{\Z}{\field{Z}}
\newcommand{\Q}{\field{Q}}
\newcommand{\F}{\field{F}}
\newcommand{\C}{\field{C}}
\newcommand{\R}{\field{R}}
\renewcommand{\det}[1]{\text{det}\3(#1\4)}
\newcommand{\ind}{\mathbb{1}}
\newcommand{\var}{\text{Var}}
\newcommand{\val}{\text{Val}}
\newcommand{\sd}{\text{SD}}
\newcommand{\cov}{\text{Cov}}
\newcommand{\pr}{\text{Pr}}
\renewcommand{\bf}[1]{\textbf{#1}}
\renewcommand{\it}[1]{\textit{#1}}
\renewcommand{\tt}[1]{\texttt{#1}}
\newcommand{\ul}[1]{\underline{#1}}
\newcommand{\mscr}[1]{\mathscr{#1}}
\newcommand{\spec}{\textbf{spec}}
\newcommand{\A}{\field{A}}
\newcommand{\liff}{\leftrightarrow}
\newcommand{\tr}[1]{\text{trace}\3(#1\4)}
\newcommand{\limf}{\lim_{n \to \infty}}
\newcommand{\3}{\left}
\newcommand{\4}{\right}
\renewcommand{\-}[1]{{}^{-#1}}
\newcommand{\up}[1]{{}^{#1}}
\newcommand{\Id}{\text{Id}}
\newcommand{\cat}[1]{\mathcal{#1}}
\newcommand{\bdiv}{\ \textbf{div} \ }
\newcommand{\bbmod}{\ \textbf{mod} \ }
\newcommand{\Hom}{\text{Hom}}
\newcommand{\Rel}{\text{Rel}}
\newcommand{\id}[1]{\text{id}_{#1}}
\newcommand{\proj}[1]{\text{proj}_{#1}}
\newcommand{\ppmod}[1]{\ (\text{mod $#1$})}
\newcommand{\ceil}[1]{\3\lceil \text{$#1$} \4\rceil}
\newcommand{\floor}[1]{\3\lfloor \text{$#1$} \4\rfloor}
\newcommand{\power}[1]{\mathcal{P}(#1)}
\newcommand{\tri}[1]{\triangle_{#1}}
\newcommand{\inj}{\hookrightarrow}
\newcommand{\im}{\text{Im}}
\newcommand{\edel}{\epsilon-\delta}
\newcommand{\empt}{ \varnothing}
\newcommand{\inv}{^{-1}}
\newcommand{\ideal}[1]{\mathfrak{#1}}
\def\upint{\mathchoice%
    {\mkern13mu\overline{\vphantom{\intop}\mkern7mu}\mkern-20mu}%
    {\mkern7mu\overline{\vphantom{\intop}\mkern7mu}\mkern-14mu}%
    {\mkern7mu\overline{\vphantom{\intop}\mkern7mu}\mkern-14mu}%
    {\mkern7mu\overline{\vphantom{\intop}\mkern7mu}\mkern-14mu}%
  \int}
\def\lowint{\mkern3mu\underline{\vphantom{\intop}\mkern7mu}\mkern-10mu\int}
\begin{document}
\title{\vspace{-1.5 cm}\bf{PredictionIO: Modeling Text Data Engine}}
\author{\vspace{-2cm}}
\date{}
\maketitle


In the real world, there are many applications that collect text as data. For example, suppose that you have a set of news articles and you want to implement an automatic categorization system that groups existing articles based on content similarity, and assigns future news articles into one of these categories. There are a wide array of machine learning models you can use to first cluster the news articles into categories and subsequently build a predictive model to classify new articles into the learned categories. However, before being able to use these techniques you must first transform the text data (in this case the set of articles) into numeric vectors, or feature vectors, that can be used to create, or train, a model.

The purpose of this tutorial is to illustrate how you can go about doing this using PredictionIO's platform. The advantages of using this platform include distributed data processing and model training for an increase in performance speed, as well as the capacity to use a newly trained predictive model to respond to queries in real-time. In particular, we will show you how to:

\begin{itemize}
\item[$\bullet$]{import a corpus of text documents into PredictionIO's event server;}

\item[$\bullet$]{read the imported event data for use in text processing;}

\item[$\bullet$]{transform document text into a feature vector;}

\item[$\bullet$]{use the feature vectors to fit a Naive Bayes classification model (using Spark MLLib library implementation);}

\item[$\bullet$]{use the feature vectors to fit a Latent Dirichlet Allocation clustering model (using Spark MLLib library implementation.}

\item[$\bullet$]{evaluate the performance of the fitted models;}

\item[$\bullet$]{yield predictions to queries in real-time using a fitted model.}
\end{itemize}


\section*{Prerequisites}

Before getting started, please make sure that you have the latest version of PredictionIO installed (https://docs.prediction.io/install/). You will also need PredictionIO's Python SDK (https://github.com/PredictionIO/PredictionIO-Python-SDK), and the Scikit learn library (http://scikit-learn.org/stable/) for importing a sample data set into the PredictionIO Event Server. Any Python version greater than 2.7 will work for the purposes of executing the \tt{data/import\_eventserver.py} script provided with this engine template.

You should also download the engine template named Modeling Text Data (http://templates.prediction.io/) that accompanies this tutorial.

\section*{Engine Overview}

The engine follows the general DASE architecture which we briefly review here. Firstly, as a user, you are charged with collecting data from your web or application, and importing it into PredictionIO's Event Server. Once the data is in the server, it  can be read and processed by our engine via the DataSource and Preparation components, respectively. The Algorithm engine component then trains a predictive model using the processed, or prepared, data. Once we have trained a model, we are ready to deploy our engine and respond to real-time queries via the Serving component. The Evaluation component is used to compute an appropriate metric to test the performance of a fitted model, as well as aid in the tuning of model hyper parameters. 

In addition to the DASE components, our engine also includes the components DataModel and TrainingModel. DataModel is the muscle in the Preparator stage as it is the component that vectorizes the text data. The TrainingModel component which is more of a conceptual framework representing a set of Scala classes that can produce a predictive model. The two particular Scala classes implemented in the engine template are named SupervisedModel and UnsupervisedModel. The figure below shows a graphical representation of the engine architecture just described, as well as its interactions with your web/app and a provided Event Server.

\vspace{0.15cm}
\centerline{
\xymatrixcolsep{4pc}\xymatrix{
\encircled{\bf{Evaluation}} & \ar[l] \\
\encircled{\bf{Event Server}} \ar[r] & \\
& \ar[ldd] \\
& \\
\encircled{\bf{Your Web/App}} \ar[ruu] \ar[uuu]^{\text{Text Data}}& 
}
\rected{
\xymatrixcolsep{2pc}\xymatrix{
\encircled{Data Source}\ar[r]& \encircled{Preparator} \ar[r] \ar[d]& \encircled{Training Algorithm} \ar[d] \ar[r] & \encircled{Serving} \\
\bf{Engine} & \encircled{Data Model} \ar[ru] & \encircled{Training Model} \ar[ru]&
}}}

\section*{Importing Data}

In order to stick with the news article example, we will be importing two different sources of data into PredictionIO's: a corpus of news documents that are categorized into a set of topics, as well as a set of stop words. Stop words are words that we do not want to include in our corpus when modeling our text data. For the remainder of the tutorial, we will assume that the present working directory is the engine template root directory.

The script in the  used to import the data is named \tt{import\_eventserver.py}. To actually import the data into our Event Server, we must first create an application. To do this run the shell command \tt{pio app new MyTextApp}, and take note of your access key. If you forget your access key, you can obtain it by using the command \tt{pio app list}. Now, use the command:
$$
\tt{python import\_eventserver.py --access\_key **** --url http://...}
$$
where you replace \tt{*****} with your actual access key. You generally should not have to specify the url argument, but if you need to just follow the same structure as above. If the data is successfully imported, you should see the following output:

\begin{verbatim}
 Importing data.....
 Imported 11314 events.
 Importing stop words.....
 Imported 318 stop words.
 \end{verbatim}
 
Our data is now imported

\break

\section*{Engine Components}

\subsection*{Data Source}

\subsection*{Preparator}

\subsection*{Data Model}

Our data model implementation is actually just a Scala class taking in as parameters \tt{td}, \tt{nMin}, \tt{nMax}, where \tt{td} is an object of class \tt{TrainingData}, and the other two parameters are the components of our n-gram window which we will define shortly. In this section, we give an overview of how we go about representing our document strings. It will be easier to explain this process with an example, so consider the document:
$$
D := \tt{"Hello, my name is Marco."}
$$
The first thing we need to do is break up $D$ into a list of \enquote{allowed tokens.} You can think of a token as a terminating sequence of characters that exist in our document (think of a word in a sentence). For example, the list of tokens that appear in $D$ is:
$$
\tt{Hello} \to \tt{,} \to \tt{my} \to \tt{name} \to \tt{is} \to \tt{Marco} \to \tt{.}
$$
Now, recall that when we imported our data, we also imported a set of stop words. This set of stop words contains all the words (or tokens) that we do not want to include once we tokenize our documents. Hence, we will call the tokens that appear in $D$ and are not contained in our set of stop words allowed tokens. So, if our set of stop words is $\{\tt{my}, \tt{is}\},$ then the list of allowed tokens appearing in $D$ is:
$$
\tt{Hello} \to \tt{,} \to \tt{name} \to \tt{Marco} \to \tt{.}
$$



\end{document}